\chapter{Die Navier-Stokes-Gleichungen im kritischen Raum $\Ell^\infty(0,T; \Ell^3_\sigma(\Omega))$}

Sei $\Omega \subset \R^3$ und $T > 0$.
Betrachte die inkompressiblen Navier-Stokes-Gleichungen
$$
\text{(NST)} \begin{cases}
  \partial_t u - \Delta u + (u \cdot \nabla) u + \nabla \pi &= 0, \quad 0 < t < T, x \in \Omega \\
  \div u &= 0, \quad 0 < t < T, x \in \Omega \\
  u &= 0, \quad 0 < t < T, x \in \partial\Omega \\
  u(0) &= a, \quad x \in \Omega.
\end{cases}
$$
Hierbei fordern wir als Kompatibilitätsbedingung, dass $a$ im richtigen Sinne "divergenzfrei" sei.

Betrachten wir nun fürs $\Omega = \R^3$, $T = \infty$ und $\lambda > 0$ die Skalierungen
\begin{align*}
  u_\lambda(x,t)  &\coloneqq \lambda u(\lambda x, \lambda^2 t) \\
  \pi_\lambda(x,t) &\coloneqq \lambda^2 \pi(\lambda x, \lambda^2 t).
\end{align*}
Man überzeugt sich leicht davon, dass falls $u$,$\pi$ das System (NST) lösen, so lösen auch $u_\lambda$, $\pi_\lambda$ dieses System.

Weiterhin gilt
\begin{align*}
  \sup_{t > 0} \Big( \int_{\R^3} |u_\lambda(x,t)|^3 \d x \d x^{\frac{1}{3}} 
  &= \sup_{t > 0} \Big( \int_{\R^3} |\lambda u (\lambda x, \lambda^2 t) |^3 \d x \Big)^{\frac{1}{3}} 
  = \sup_{t > 0} \Big( \int_{\R^3} |u(y,t)|^3 \d y \Big)^{\frac{1}{3}},
  \end{align*}
also ist die Norm von $\Ell^\infty(0,\infty; \Ell_\sigma^3(\R^3))$ invariant unter dem natürlichen Skalieirungsverhalten der Navier-Stokes-Gleichungen.
Wir wollen den Raum $\Ell^\infty(0,\infty; \Ell^3_\sigma(\R^3))$ als \emph{kritischen Raum} bezeichnen.

Angenommen, man könnte zeigen, dass für den Lifespan $T_0$ von $u$ eine Abschätzung
\begin{align*}
  \tag{\$} T_0 \geq C(\|a\|_{\Ell^3})
\end{align*}
existiert, wobei $C(\|a\|_{\Ell^3}) > 0$ im Wesentlichen von $\|a\|_{\Ell^3}$ abhängt.
Definiere nun für $\lambda > 0$
$$
a_\lambda(x) \coloneqq \lambda a(\lambda x),
$$
so folgt $\|a_\lambda\|_{\Ell^3} = \|a\|_{\Ell^3}$ und für jedes $\lambda > 0$ existiert $u_\lambda$ mindestens auf dem Zeitintervall $[0,T_0]$ ($T_0$ ist $\lambda$-unabhängig).
Hieraus ergibt sich, dass $(u_\lambda)_{\frac{1}{\lambda}}$ mindestens auf dem Zeitintervall $[0,\lambda^2 T_0]$ existiert und es gilt
\begin{align*}
  \tag{$\ast$} \|u\|_{\Ell^\infty(0,\lambda^2 T_0; \Ell^3_\sigma(\R^3))} = \|u_\lambda\|_{\Ell^\infty(0,T_0; \Ell^3_\sigma(\R^3))}.
\end{align*} 
Hätte man nun noch auf dem Lifespan-Intervall eine geeignete Abschätzung der Lösung gegen die Daten, d.h.
\begin{align*}
  \tag{\$\$} \|u\|_{\Ell^\infty(0,T_0; \Ell^3_\sigma(\R^3))} \leq C(\|a\|_{\Ell^3}),
\end{align*}
so folgt mit ($\ast$)
$$
\|u\|_{\Ell^\infty(0,\lambda^2 T_0; \Ell^3_\sigma(\R^d))}
\overset{\text{($\ast$)}}{=} \|u_\lambda\|_{\Ell^\infty(0,T_0; \Ell^3_\sigma(\R^3))}
\overset{\text{(\$\$)}}{\leq} C(\|a_\lambda\|_{\Ell^3}) = C(\|a\|_{\Ell^3})
$$
und für $\lambda \to \infty$ ergäbe sich $u \in \Ell^\infty(0,\infty; \Ell^3_\sigma(\R^3))$, wir hätten damit also aus einer lokalen Lösung eine globale Lösung gemacht.

Wie bekommt man jetzt die Million?
Falls die Anfangsdaten $a \in \Ell^2_\sigma(\R^3) \cap \Ell^3_\sigma(\R^3)$ und $v$ "schwache Lösung" von (NST) die "Energieungleichung" erfüllt, so haben Kozono und sohr 1996 gezeigt dass dann $u = v$ gilt.

Wenn $a$ zusätzlich eine Schwartz-Funktion ist und die u "schwache Lösung" in $\Ell^\infty(0,\infty,\Ell^3_\sigma(\R^3))$, so haben Escramiaza, Seregin Sverak 2003 gezeigt, dass in diesem Falle $u$ glatt ist in $[0,\infty) \times \R^3$. Somit gilt
$$
\text{Zeige (\$) und (\$\$)} \implies \text{Millionär}
$$

Wir beschäftigen uns im Folgenden mit der Lösbarkeit von (NST) in $\Ell^\infty(0,T; \Ell^3_\sigma(\Omega))$, wobei $\Omega \subset \R^3$ ein beschränktes Lipschitz-Gebiet ist.
Genauer geht es um lokale Lösbarkeit in der Zeit für beliebig große Anfangsdaten und globale Lösbarkeit für kleine Anfangsdaten. Eine formale Anwendung der Helmholtz-Projektion auf (NST) liefert
$$
\begin{cases}
  \partial_t u + Au & = - \PP(u\cdot \nabla u) \\
  u(0) &= a.
\end{cases}
$$
Sieht man die Nichtlinearität als rechte Seite, so müsste $u$ durch die Variation-der-Konstanten-Formel
$$
u(t) = \e^{-tA} a - \int_0^t e^{-(t - s)A_p}(u(s) \cdot \nabla) u(s) \d s
$$
gegeben sein. 
Dies motiviert die folgende Definition:
\begin{defn}
  Seien $0 < T \leq \infty$, $r \geq 3$ und $a \in \Ell_\sigma^r(\Omega)$.
  Dann heißt $u \colon [0,T) \to \Ell_\sigma^r(\Omega)$ \emph{milde Lösung} von (NST) mit  Anfangswert $a$, falls $u \in \CC(0,T; \Ell_\sigma^r(\Omega))$ und für alle $0 < t< T$ und ein $p \geq r$
  $$
  \big( s \mapsto \e^{-(t - s)A} \PP (u(s) \cdot \nabla) u(s) \big) \in \Ell^1(0,t; \Ell_\sigma^p(\Omega))
  $$
  und
  $$
  u(t) = \e^{-t A} a - \int_0^t \e^{-(t- s) A} \PP(u(s) \cdot \nabla )u(s) \d s.
  $$
\end{defn}

\begin{hsatz}
  \label{hsatz:momomoney}
  Sei $\Omega \subset \R^d$ ein beschränktes Lipschitz-Gebiet.
  Dann existiert ein $\varepsilon > 0$, sodass für alle $3 \leq r < 3 + \varepsilon$ und alle $a \in \Ell_\sigma^r(\Omega)$ die folgenden Aussagen gelten.
  \begin{enumerate}[i)]
    \item Es existiert $T_0 > 0$ und eine milde Lösung von $u \colon [0,T_0) \to \Ell_\sigma^r(\Omega)$ von (NST) mit Anfangswert $a$, sodass für alle $r \leq p < 3 + \varepsilon$ mit $\frac{3}{2} \big( \frac{1}{r} - \frac{1}{p} \big) < \frac{1}{4}$ gilt
      \begin{align*}
        &\big( t \mapsto t^{\frac{3}{2}(\frac{1}{r} - \frac{1}{p})} u(t) \big) \in \mathrm{BC}([0,T_0), \Ell^p_\sigma(\Omega)) \\
        &\big( t \mapsto t^{\frac{1}{2} + \frac{3}{2}(\frac{1}{r} - \frac{1}{p})} \nabla u(t) \big) \in \mathrm{BC}([0,T_0), \Ell^p(\Omega; \C^9) 
      \end{align*}
      Weiterhin gilt
      $$
      \|u(t) - a \|_{\Ell^r} \to 0 \quad\text{für } t \searrow.
      $$
      Falls $r < p < 3 + \varepsilon$ gilt, dass 
      $$
      t^{\frac{3}{2}(\frac{1}{r} - \frac{1}{p})} \|u(t)\|_{\Ell^p} \to 0 \quad\text{für } t \searrow 0
      $$
      und, falls $r \leq p < 3 + \varepsilon$ gilt, so folgt
      $$
      t^{\frac{1}{2} + \frac{3}{2}\big( \frac{1}{r} - \frac{1}{p} \big)} \|\nabla u (t) \|_{\Ell^p} \to 0 \quad\text{für } t \searrow 0.
      $$

    \item Falls $r > 3$, so existiert $C > 0$, sodass $T_0 \geq C \cdot \|a\|_{\Ell^r}^{-\frac{2r}{r - 3}}$.
    \item Für alle $3 \leq p < 3 + \varepsilon$ existieren $C_1, C_2 > 0$, sodass unter der Voraussetzung, dass $\|a\|_{\Ell^3} \leq C_1$, die milde Lösung global ist, d.h. $T_0 = \infty$.
      Außerdem gelten die Abschätzungen für das Langzeitverhalten
      \begin{align*}
        \|u(t)\|_{\Ell^p} &\leq C_2 \cdot t^{\frac{3}{2p} - \frac{1}{2}}, \quad\text{für alle } 0 < t < \infty \\
        \|\nabla u(t) \|_{\Ell^p} &\leq C_2 \cdot t^{\frac{3}{2p} - 1}, \quad\text{für alle } 0 < t < \infty.
      \end{align*}
  \end{enumerate}
\end{hsatz}
Für den Beweis von Hauptsatz \ref{hsatz:momomoney} definieren wir iterativ
\begin{align*}
  u_0(t) &\coloneqq \e^{-t A} a \\
  u_{j + 1}(t) &\coloneqq u_0(t) - \int_0^t \e^{-(t - s) A} \PP(u_j(s) \cdot \nabla) u_j(s) \d s \quad\text{für alle } j \in \N_0.
\end{align*}
Für $p \geq r$ definiere weiterhin $\sigma \coloneqq \frac{3}{2} \big( \frac{1}{r} - \frac{1}{p} \big)$ und für $T > 0$ definieren wir die Größen
\begin{align*}
  K_j &\coloneqq K_j(T) \coloneqq \sup_{0 < t < T} t^\sigma \|u_j(t) \|_{\Ell^p} \\
  R_j &\coloneqq R_j(T) \coloneqq \sup_{0 < t < T} t^{\frac{1}{2} + \sigma} \|\nabla u_j(t) \|_{\Ell^p}.
\end{align*}
Der Parameter $\sigma$ stammt aus Satz \ref{thm:lplqSmoothing}.

\begin{proof}[Beweis von Hauptsatz \ref{hsatz:momomoney}]
  Wir werden uns hier nicht um die Stetikgeit oder Messbarkeit der Integranden kümmern. Dies kann aber per Induktion als Übungsaufgabe bewiesen werden.

  \textbf{Schritt 1}: Wir zeigen zunächst die  Beschränktheit der Folgen $K_j$ und $R_j$ für $3 \leq r < p < 3 + \varepsilon$ und $3 < r \leq p < 3 + \varepsilon$.
  Mit Satz \ref{thm:lplqSmoothing} folgt zunächst
  \begin{align*}
    \|u_0(t) \|_{\Ell^p} &= \|\e^{-t A} a \|_{\Ell^p} \leq C \cdot t^{-\sigma} \|a\|_{\Ell^r} \\
    \|\nabla u_0(t) \|_{\Ell^p} &= \|\nabla \e^{-\frac{t}{2} A} \e^{-\frac{t}{2} A} a \|_{\Ell^p} \leq C\cdot t^{-\frac{1}{2}} \|\e^{-\frac{t}{2} A} a \|_{\Ell^p} \leq C\cdot t^{-\frac{1}{2}   - \sigma} \|a\|_{\Ell^r}.
  \end{align*}
  Dies zeigt $K_0, R_0 < \infty$.

  Sei $0 < t < T$. 
  Nehme induktiv and, dass $K_j, R_j < \infty$.
  Dann folgt
  \begin{align*}
    \|u_{j + 1}(t) \|_{\Ell^p}
    &\leq \|u_0(t)\|_{\Ell^p} + C \, \int_0^t \|\e^{-(t - s) A} \PP(u_j(s) \cdot \nabla) \\
    &\leq \|u_0(t)\|_{\Ell^p} + C \, \int_0^t (t - s)^{\frac{3}{2} (\frac{2}{p} - \frac{1}{p})} \|u_j(s) \cdot \nabla) u_j(s) \|_{\Ell^{\frac{p}{2}}} \d s \\
    &\leq \|u_0(t)\|_{\Ell^p} + C \, \int_0^t (t - s)^{\frac{3}{2p}} s^{-2\sigma - \frac{1}{2}} \d s \, K_j \cdot \R_j,
  \end{align*}
  wobei wir zunächst Satz \ref{thm:lplqSmoothing} und Haupsatz \ref{hsatz:contProj} anwenden und danach Höldern sowie die Induktionsvoraussetzung nutzen.
  Weiterhin ist
  $$
  \frac{3}{2p} < \frac{1}{2} \quad\text{und}\quad 2\sigma + \frac{1}{2} < 1,
  $$
  da nach Voraussetzung $\sigma < \frac{1}{4}$ gilt.
  Daraus folgt mit der Substitution $s = xt$
  $$
  \int_0^t (t - s)^{-\frac{3}{2p}} s^{-2\sigma - \frac{1}{2}} \d s = t^{1 - \frac{3}{2p} - 2\sigma - \frac{1}{2}} \int_0^1 ( 1 - x )^{-\frac{3}{2p}}  x^{- 2\sigma - \frac{1}{2}} \d x.
  $$
  Und da
  $$
  \frac{1}{2} - \frac{3}{2p} - \sigma = \frac{1}{2} - \frac{3}{2r}
  $$
  gilt folgt
  $$
  K_{j + 1} \leq K_0 + C\; T^{\frac{1}{2} - \frac{3}{2r}} K_j R_j.
  $$
  Für den Gradienten gilt mit analoger Argumentation
  \begin{align*}
    \|\nabla u_{j + 1}(t) \|_{\Ell^p}
    &\leq \|\nabla u_0 (t) \|_{\Ell^p} + \int_0^t \|\nabla \e^{-(t - s) A} \PP(u_j(s) \cdot \nabla) u_j(s) \|_{\Ell^p} \d s \\
    &\leq \|\nabla u_0(t) \|_{\Ell^p} + C \int_0^t (t - s)^{-\frac{1}{2}} \; \|\e^{-\frac{1}{2}(t - s)A} \PP(u_j(s) \cdot \nabla) u_j(s) \|_{\Ell^p} \d s \\
    &\leq \|\nabla u_0(t) \|_{\Ell^p} + C \int_0^t (t - s)^{-\frac{1}{2} - \frac{3}{2p}} \|(u_j(s) \cdot \nabla) u_j(s) \|_{\Ell^{\frac{p}{2}}} \d s
  \end{align*}
\end{proof}
