\chapter{Die Ungleichung von Gagliardo-Nirenberg}

\begin{ntion}
  Sei $-\infty < \frac{1}{p} \leq 1$.
  Fall $0 \leq \frac{1}{p} \leq 1$, dann definiere
  $$
  \| \,\cdot\, \|_{X_{\frac{1}{p}}} \coloneqq \| \,\cdot\, \|_{\Ell^p}
  $$
  und falls $-\infty < \frac{1}{p} < 0$ sei $\alpha \in [0,1)$ und $k \in \N_0$ derart, dass $-\frac{d}{p} = k + \alpha$. Definiere
  $$
  \| \,\cdot\, \|_{X^{\frac{1}{p}}} \coloneqq 
  \begin{cases} 
    \| \nabla^k \cdot \|_{\Ell^\infty}, \quad &\alpha = 0, \\ [ \nabla^k \, \cdot \, ]_\alpha, \quad &\alpha \neq 0,
  \end{cases}
  $$
  wobei $[\,\cdot\,]_\alpha$ die Hölder-Halbnorm zum Exponenten $\alpha$ bezeichne.
\end{ntion}

\begin{hsatz}[Gagliardo-Nirenberg]
  \label{hsatz:gagliardoNirenberg}
  Seien $1 \leq q,r < \infty$, $d \geq 2$ und $j,m \in \N_0$ mit $0 \leq j \lneq m$.
  Weiterhin sei
  $$
  \begin{cases}
    \frac{j}{m} \leq \alpha \leq 1, \quad \text{falls } m - j - \frac{d}{r} \not\in \N_0 \\
    \frac{j}{m} \leq \alpha < 1, \quad \text{falls } m - j - \frac{d}{r} \in \N_0.
  \end{cases}
    $$
    und
    $$
    \frac{1}{p} \coloneqq \frac{j}{d} + \alpha\, \Big( \frac{1}{r} - \frac{m}{d} \Big) + (1 - \alpha) \, \frac{1}{q}.
    $$
    Dann ist $\frac{1}{p} \leq 1$ und es existiert eine Konstante
    $$
    C = C(d,m,j,q,r,\alpha) > 0,
    $$
    sodass für alle $u \in \CC_c^m(\R^d)$ gilt
    $$
    \| \nabla^j u\|_{X_{\frac{1}{p}}} \leq C \, \|\nabla^m u\|_{\Ell^r}^\alpha \, \|u\|_{\Ell^q}^{1 - \alpha}.
    $$
\end{hsatz}

Für den Beweis benötigen wir einige Vorbetrachtungen.

\begin{lem}
  \label{lem:hoelderIneq}
  Sei $r > d \geq 2$.
  Dann existiert $C = C(d,r) > 0$, sodass für alle $u \in \CC_c^1(\R^d)$ und $x,y \in \R^d$ gilt
  $$
  \frac{\left| u(x) - u(y) \right|}{|x - y|^{1 - \frac{d}{r}}} < C \, \|\nabla u\|_{\Ell^r}.
  $$
\end{lem}

\begin{proof}
  Sei $\delta \coloneqq | x - y|$ und $\BB \coloneqq \BB(x,\delta) \cap \BB(y,\delta)$.
  Dann gilt
  $$
  |u(x) - u(y)| \cdot |\BB|
  \leq \int_{\BB} |u(x) - u(z)| \d z + \int_{\BB} |u(z) - u(y)| \d z.
  $$
  Anwendung des Hauptsatzes liefert für das erste Integral
  \begin{align*}
    \int_\BB |u(x) - u(z)| \d z 
    &\leq \int_{\BB(x,\delta)} \int_0^{|x - z|} | \frac{\d{}}{\d t} [u(x + t \frac{z - x}{|z - x|})] \d t \d z \\
    &= \int_{\BB(0,\delta)} \int_0^{|z'|} \Big| \frac{\d{}}{\d t} \big[\, u(x + t \frac{z'}{|z'|} ) \, \big] \Big| \d t \d z' \\
    &= \int_{\partial\BB(0,1)} \int_0^\delta \Big( \int_0^\rho \Big| \frac{\d{}}{\d t} u(x + t \omega) \Big| \d t \Big) \; \rho^{d - 1} \d \rho \d \sigma(\omega)  \\
    &= \int_{\partial\BB(0,1)} \int_0^\delta \Big( \int_t^\delta \rho^{d - 1} \d \rho \; \Big) \Big| \frac{\d{}}{\d t} \big[ u(x + t \omega) \big] \Big| \d t \d \sigma(\omega) \\
    &\leq \frac{\delta^d}{d} \int_{\BB(0,\delta)} |z'|^{1 - d} |\nabla u(x + z')| \d z' \\
    &\leq \frac{\delta^d}{d} \Big( \int_{\BB(0,\delta)} |z'|^{\frac{r(1 - d)}{r - 1}} \d z' \Big)^{\frac{r - 1}{r}} \, \|\nabla u\|_{\Ell^r(\BB(x,\delta))} \\
    &= \sigma(\BB(0,1))^{\frac{r - 1}{r}} \frac{\delta^{d + 1 - \frac{d}{r}}}{d(d + \frac{r(1 - d)}{r - 1} )^{\frac{r - 1}{r}}} \, \| \nabla u\|_{\Ell^r(\BB(x,\delta))}
  \end{align*}
  Aus Symmetriegründen folgt
  $$
  \int_{\BB} |u(z) - u(y)| \d z
  \leq C \delta^{d + 1 - \frac{d}{r}} \|\nabla u\|_{\Ell^r(\BB(y,\delta))}.
  $$
  Weiterhin folgt aus $\BB(\frac{1}{2}(x + y), \frac{\delta}{2}) \subset \BB$, dass $|\BB| \geq |\BB(0,1)| 2^{-d} \delta^d$.
  Hieraus ergibt sich 
  $$
  |u(x) - u(y)| \delta^d \leq C \delta^{d + 1 - \frac{d}{r}} \| \nabla u\|_{\Ell^r(\R^d)},
  $$
  wobei $C = C(d,r)$.
\end{proof}

Das folgende Lemma reduziert den Beweis von Haupsatz \ref{hsatz:gagliardoNirenberg} auf wenige Spezialfälle.

\begin{lem}
  \label{lem:reducingGagliardo}
  \begin{enumerate}[a)]
    \item Angenommen die Ungleichung in Haupsatz \ref{hsatz:gagliardoNirenberg} gelte für $\alpha = \frac{j}{m}$ mit $j = 1$ und $m = 2$, dann gilt die Ungleichung auch für $\alpha = \frac{j}{m}$ und jedes $0 \leq j < m$.

    \item Angenommen die Ungleichung in Haupsatz \ref{hsatz:gagliardoNirenberg} gelte für $\alpha = 1$, $j = 0$ und $m = 1$ (wobei $d \neq r$), dann gilt die Ungleichung auch für $\alpha = 1$ und jedes $0 \leq j < m$ vorausgesetzt $m - j - \frac{d}{r} \not\in \N_0$.
      
    \item Für alle $-\infty < \lambda \leq \mu \leq \nu \leq 1$ existiert $C = C(\lambda, \mu, \nu) > 0$, sodass für alle $f \in X_\nu \cap X_\lambda$ die sogennante Interpolationsungleichung
      $$
      \| f\|_{X_\mu} \leq C \, \|f\|_{X_\lambda}^{\frac{\nu - \mu}{\nu - \lambda}} \, \|f\|_{X_\mu}^{\frac{\mu - \lambda}{\nu - \lambda}}
      $$
      gilt.
      Insbesondere ist $f \in X_\mu$.

    \item Angenommen die Ungleichung in Haupsatz \ref{hsatz:gagliardoNirenberg} gelte für $\alpha = \frac{j}{m}$ und $\alpha = 1$, dann gilt diese auch für jedes $\frac{j}{m} \leq \alpha \leq 1$.
  \end{enumerate}
\end{lem}

\begin{proof}
  Übung.
\end{proof}

Nun sind wir in der Lage Haupsatz \ref{hsatz:gagliardoNirenberg} zu beweisen.

\begin{proof}[Beweis von Haupsatz \ref{hsatz:gagliardoNirenberg}]
  Dass $\frac{1}{p} \leq 1$ gilt, ist Übungsaufgabe.
  Lemma \ref{lem:reducingGagliardo} reduziert den Beweis auf die folgenden Fälle
  \begin{itemize}
    \item $\alpha = 1$, $j = 0$, $m = 1$ (für $r \neq d$).
    \item $\frac{j}{m} < \alpha < 1$ und $m - j - \frac{d}{r} \in \N_0$.
    \item $\alpha = \frac{1}{2}$, $j = 1$, $m = 2$.
  \end{itemize}

  \begin{enumerate}[\text{Fall} 1:]
    \item $\alpha = 1$, $j = 0$, $m = 1$.
      Es gilt $\frac{1}{p} = \frac{1}{r} - \frac{1}{d}$.
      Sei erst $r > d$, und damit $\frac{1}{p} < 0$ und
      $$
      -\frac{d}{p} = 1 - \frac{d}{r} \quad\text{(Hölder-Exponent zu $X_{\frac{1}{p}}$)}
      $$
      Dann folgt die Behauptung aus Lemma \ref{lem:hoelderIneq}.
      Sei nun $r = 1 < d$, $x \in \R^d$, $1 \leq i \leq d$ und $\gamma_i \colon (-\infty, \infty) \to \R^d$ definiert durch $\gamma_i (t) \coloneqq x \cdot t e_i$.
    \item 
  \end{enumerate}

\end{proof}

